\chapter{Crypto cont.}
More notes on one-way permutations: a function that grows slower than all inverse polynomials is called \emph{negligible}---as a probability, it means something is basically impossible.

\section{Rabin's function}
We will not give a one-way function that is not one-to-one but four-to-one: the Rabins function.
For a size \(n\), choose \(N\) with \(n\) bits. Then the mapping is \(x^2 \mod N\).

We assume that factoring (input \(N\), output \((p, q)\)) is hard. Can this show that Rabin's problem (input \(x^2 \mod N\) with \(N = pq\), output \(x\)) is hard?

\subsection{{Factoring} \(\to\) {Rabin's}}
Use the same \(N\) for both instances. Sample \(z\) at random and give \(y =  z^2\mod N\) to the \textsc{Rabin's} solver.
The solver will output \(\pm z\) or \(\pm z'\) (the other square root) w.p.~\(\frac{1}{2}\epsilon\).
Then \(N\) is factored as \(\left(z + z'\right)\left(z - z'\right)\) (you can get primes using GCD).\footnote{There is no way for an adversary to provide a bad \textsc{Rabin's} solver (that systematically misses a square root pair), because the reduction algorithm randomly chooses either square root of \(z^2\).} The same algorithm can be made one-to-one by restricting the choices of domain and \(N\).

\section{Hardness-backed psuedorandomness}
Suppose we have a one-way permutation \(f: 2^n \to 2^n\).
If we pick a random \(x\) and apply \(f\) to it, the result will be uniformly distributed (because \(f\) is bijective).
We will imagine an algorithm that takes \(2n\) bits of randomness and outputs \(2n+1\) bit of randomness:
\begin{align}
x, r &\mapsto f(x), r, B(x,r) \\
B(x, r) &= \left\langle x_i, r_i\right\rangle \mod 2
\end{align}
\(B\) is the hardcore bit function.
\begin{theorem}
	For any PPTA \(A\), \(\Pr\left[ A\left(f(x), r\right) = B(x, r)\right] - \frac{1}{2}\) is negligible.
\end{theorem}
\begin{proof}
	We will show that computing this one bit is as hard as inverting the entire function.
	We will find a reduction from one-way-inversion to hardcore-bit-prediction. (Sadly, it is only a Turing reduction.)
	
	In the case that \(A\) is \emph{certainly} right, it suffices to vary \(r\) over a basis of \(\left(\mathbb{Z}/2\mathbb{Z}\right)^n\).
	For unreliable or tricky \(A\) functions you will have to try more vectors more times to hit a basis that reconstructs \(x\) from \(f(x)\).
\end{proof}

Given a OWP \(f: 2^n \to 2^n\), is \(g: 2^{m \gg n} \to 2^{m \gg n}\) that does \(f\) on the first \(n\) bits and then passes all the rest through an OWP?
\begin{theorem}
	(See above question.) Yes.
\end{theorem}
\begin{proof}
	Reduction again. Given \(y = f(x)\), we will pass \(y\oplus r\) (\(r\) a random string) to \(g\), and we will from the former bits recover \(x\).
\end{proof}

\section{Coin flipping over the phone}
Alice and Bob are flipping a coin to decide who will pay for dinner: one person will flip a coin and the other will reports the assignment.
But this can be solved with hardcore bits.

\begin{enumerate}
	\item \textsc{Alice} samples \(x\) and \(r\) from \(2^n\).
	\item \textsc{Alice} sends \(f(x), r\) to \textsc{Bob}. (\textsc{Alice} knows \(B(x,r)\) but \textsc{Bob} does not.)
	\item \textsc{Bob} reports an assignment \(c\in 2\).
	\item \textsc{Alice} and \textsc{Bob} agree on \(c\oplus B(x, r)\). (\textsc{Bob} is able to verify \(B(x,r)\) because he knows \(f\) as well.)
\end{enumerate}
