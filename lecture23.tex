\chapter{Even more NP-completeness}
\section{ZOE \(\to\) ILP}
The reduction is immediate by expressing equality constraints by inequalities.
\section{ZOE \(\to\) Subset Sum}
ZOE is \(Ax = 1\). Subset sum is given \(\left\{a_1,\ldots,a_m\right\}\) to find \(S\subseteq m\) such that \(\sum_{i\in S} a_i = B\).

One idea is to represent the numbers as binary numerals in columns of \(A\). But combining them using \(x\) doesn't work because there will be no carry in the addition.
But the solution is to represent in a higher radix? So carry is not necessary.

\section{ZOE \(\to\) Rudrata Cycle}
We will first reduce ZOE to \textsc{Rudrata Cycle with Paired Edges}.
For RCPE, you have a set of edge pairs \(C \subseteq E\times E\) and are
allowed to use only one edge from each pair. (You are allowed to have more than one edge per vertex pair.)

We will try to convert ZOE into a graph. The idea is that for each
component of \(x\) we will make a gadget of two vertices joined by two edges,
corresponding to 0 or 1.
Each row of \(A\) (dotted with \(x\)) will correspond to a two-vertex graph joined by edges labeled according to the positions of \(1\) in the row.
Connect all the 2-vertex gadgets end to end.
The paired-edge constraints need to ensure that if several edges are labeled by the same binary variable, they need to be chosen altogether or not at all in the \textsc{rcpe} output. This is accomplished by including in \(C\) all pairs \((\lnot x, x)\), with labels taken  from the \(x\) gadget and the \(A\)-row gadgets, respectively.

Now we will reduce \textsc{rcpe} to \textsc{rc}, beginning with the case
\(\left|C\right| = 1\) (you can repeat this as many times as necessary for larger \(C\)).
Given \(e = \left(a, b\right)\) and \(e' = \left(c, d\right)\), delete these two connections and connect \(a\), \(b\), \(c\), and \(d\) using a \(3\times 4\)
grid of extra vertices and edges (too hard to draw)---as it happens, the only way to go from \(a\) to \(b\) is to take up vertices that would otherwise be required for connecting \(c\) to \(d\).

\section{Approximation algorithm}
Suppose we are trying to minimize something.
The approximation ratio \(\alpha_A\) for an algorithm \(A\) is defined as
\(\max_{\text{instance}\ I} \frac{A(I)}{\operatorname{optimal}(I)}\).

There is an approximation for \textsc{Vertex Cover} with \(\alpha = 2\).
A \emph{maximum matching} is a matching for which no more edges can be added.
\begin{theorem}
	For any maximal matching \(M\) (which is a vertex cover), \(2\left|M\right| \geq \left|\text{minimal vertex cover}\right| \geq \left|M\right|\).
\end{theorem}
\begin{proof}
	The lower bound is obvious: each vertex must be touched at least once.
	For the upper bound, assume \(S\) is not a vertex cover. Then just add the rest of the edges as needed.
\end{proof}