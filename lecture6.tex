\chapter{More graphs}
\section{Edge classification cont.}
It's interesting to describe stack behavior chronologically using the Dyck language. For instance,

\begin{description}
	\item[forward/tree edge] (tree if there is nothing between \(u\) and \(v\))
	\begin{align}
	\overset{{u}}{(}
	\quad
	\overset{{v}}{(}
	\quad
	\overset{{v}}{)}
	\quad
	\overset{{u}}{)}
	\end{align}
	
	\item[back edge]
	\begin{align}
	\overset{{v}}{(}
	\quad
	\overset{{u}}{(}
	\quad
	\overset{{u}}{)}
	\quad
	\overset{{v}}{)}
	\end{align}
	
	\item [cross edge]
	\begin{align}
	\overset{{u}}{(}
	\quad
	\overset{{u}}{)}
	\quad
	\overset{{v}}{(}
	\quad
	\overset{{v}}{)}
	\end{align}
\end{description}

\section{Directed cycles}
\begin{align}
	\exists(\text{back edge}) \iff \exists(\text{directed cycle})
\end{align}

\section{Topological sort}
The task is to find a total ordering of a DAG's vertices that is a superset of the partial order induced by \(\rightarrow\).

\subsection{Na\"ive algorithm}
Find a source vertex and remove it. This is horrible, \(\Theta\left(n^3\right)\).

\subsection{Better algorithm}
Observe that \(u\to v\), \(\text{post}(u) > \text{post}(v)\). Sort the vertices in decreasing post values. More simply, return the reverse postorder.

\section{Strongly connected components}
A graph is \emph{strongly connected} if for all \(u,v \in V\) \(\exists u\rightsquigarrow v\) and \(\exists v\rightsquigarrow v\).
A \emph{strongly connected component} is a maximal component that is strongly connected.
Thus, every graph is a DAG of strongly connected components.
(Simon's note: treat a directed graph as a commutative diagram depicting a small category. 
Then the strongly connected components are isomorphism classes.)

\subsection{Problem}
Given a directed graph \((V, E)\), output a list of SCCs. Here is one idea:
\begin{enumerate}
	\item Pick a vertex in a sink SCC. (bottom of the partial order lattice)
		\begin{itemize}
			\item If DFS goes from SCC1 to SCC2, the maximum post value in SCC1 is greater than the maximum post value in SCC2.
			\begin{itemize}
				\item Therefore, the vertex with the largest post value will be in the source SCC.
			\end{itemize}
		\end{itemize}
	\item DFS the sink SCC.
\end{enumerate}

More fully specified,
\begin{enumerate}
	\item DFS on \(\left(V, E_\text{reversed}\right)\) to identify a vertex in the sink SCC (viz.~the one with the largest post value).
	\item DFS from sink-strongly-connected vertex to identify the sink SCC.
	\item Delete the sink SCC, then repeat?
\end{enumerate}