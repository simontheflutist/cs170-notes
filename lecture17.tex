\chapter{Linear programming II}
Convert to a minimization problem with equalities in non-negative variables:
\begin{align}
\begin{matrix}
	\max & 4x_1 + 2x_2 \\
	\text{subj.~to} &\begin{aligned}
		3x_1 &\leq 60 \\
		3x_2 &\leq 75 \\
		3x_1 + 2x_2 &\leq 100 \\
		x_1, x_2 &\geq 0
	\end{aligned}
\end{matrix}
\end{align}
Solution:
\begin{align}
\begin{matrix}
\min & -\left(4x_1 + 2x_2\right)\\
& \begin{aligned}
3x_1 + s_1 &= 60 \\
3x_2 + s_2 &= 75 \\
3x_1 + 2x_2 + s_3 &= 100\\
x_1, x_2, s_1, s_2, s_3 &\geq 0
\end{aligned}
\end{matrix}
\end{align}

\section{Flows in graphs}
You have a directed graph \(G = \left(V, E\right)\) with weights \(c\) that state the maximum flow along an edge.
We want to get the maximum flow that can be passed through an edge, subject to edge capacity:
\begin{align}
\forall (u, v) \in E: f_{u,v} \leq c_{u, v}
\end{align}
and conservation:
\begin{align}
\forall v\in \left(V-\left\{s, t\right\}\right): \sum_{v^\flat\to v} f_{v^\flat,v} &= \sum_{v\to v^\sharp} f_{v, v^\sharp}
\end{align}
The objective flow from \(s\) to \(t\) is
\begin{align}
\max \sum_{s\to s^\sharp} f_{s, s^\sharp} \quad \text{or} \quad \max \sum _{t^\flat \to t} f_{t^\flat, t}
\end{align}
If we were to use \textsc{Simplex}, we'd begin with any flow that respects the given constraints (for example, a zero flow) and then improve it until a better solution is known. We will apply the same idea in a different algorithm specialized for \textsc{max flow}.

\subsection{Algorithm for max flow}
Let \(G\) have \(c_{u,v}\) for all \((u,v)\in E\).
\begin{algorithmic}[1]
	\While {\(\exists \,\text{path}\, s\rightsquigarrow t\)}
		\State \(P\gets\) \(s\rightsquigarrow t\) path in \(G\).
		\State \(f = \min_{e\in P} c_e\)
		\For{\(\left(u, v\right) \in P\)}
			\State \(c_{u, v}\gets c_{u, v} - f\)
			\State \(c_{v,u}\gets c_{v, u} + f\)
		\EndFor
	\EndWhile
\end{algorithmic}
\begin{itemize}
	\item Idea of \(c_{u, v}\gets c_{u, v} - f\):
	
	Allow all edges of \(P\) to be saturated by flow through the weakest link (this step is greedy---give \(P\) as much flow as possible).
	\item Idea of \(c_{v,u}\gets c_{v, u} + f\):
	
	This edge in the reverse direction gives you a chance to undo the flow you just committed to. In case you pushed too much flow through a path that should not accept it, a later greedy step will not be hindered by the new saturation. (Think about a maximum flows on individual edges: \(+1\) and \(-1\) cancel---you're not calculating exactly how much goes where.)
\end{itemize}
\begin{proof}
	Let \(\left(L, R\right)\) be an \(s\)-\(t\) cut. We will show that \(\max \sum_{s\to s^\sharp} f_{s, s^\sharp} \leq c(L, R)\) for any cut \((L, R)\).
	
	The \(s\)-\(t\) flow is given by
	\begin{align}
		\sum_{\begin{matrix}
			u\in L \\
			v \in R\\
			\end{matrix}} \left(f_{u, v} - f_{v, u}\right) & \phantom{{}={}}
		\intertext{Letting \(f_{u, v} \leq c_{u, v}\) and \(f_{v, u} = 0\),}
		&\leq \sum_{\begin{matrix}
			u\in L \\
			v \in R\\
			\end{matrix}} c_{u, v} = c(L, R)
	\end{align}
	So cuts are dual to flows: each cut is a bound on the possible max flow. We will show that there is no gap: the max flow is equal to the smallest cut.

	In the algorithm, we stopped as soon as there was no \(s\)-\(t\) path. Let \(L\) be the vertices reachable from \(s\) in the final graph and \(R\) the vertices co-reachable from \(t\). It appears that \(c(L, R)\) is the max flow. \(L\) and \(R\) are disjoint because all edges are already saturated by candidate flow.
	
	Perhaps there is an edge from \(R\) to \(L\). But there is no flow along the back edge, because there is no way for the flow to return to \(R\).
	
\end{proof}

\subsection{Runtime of max flow}
The time required for one iteration is \(O\left(\left|E\right|\right)\), and the number of iterations is \(O(F)\). In all, we need \(O\left(F\left|E\right|\right)\).