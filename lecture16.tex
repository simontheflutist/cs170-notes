\chapter{Linear programming}
Suppose you want to grow a vegetable garden. You can grow either peas or carrots.
\begin{center}
	
\begin{tabular}{l|rr}
	& Peas & Carrots \\ \hline
	Profit & \$4/bushel & \$2/bushel \\
	Land & 3 sq.~yd.~sunny & 3 sq.~yd.~shady \\
	Water & 3~units & 2~units
\end{tabular}
\end{center}

We have 70 sq.~yd.~of sunny land, 75 sq.~yd.~of shady land, and 100 units of water. How can we maximize our profit?
Say the quantity of peas we wish to plant is \(x_1\)
and the quantity of carrots we wish to plant is \(x_2\).
Then our constraints are:
\begin{align}
3x_1 &\leq 60 & 3x_2 &\leq 75 && \text{(land)} \\
3x_1 + 2x_2 &\leq 100 && && \text{(water)} \\
x_1 &\geq 0 & x_2 &\geq 0 && \text{(common sense/rules)}
%\end{align}
\intertext{and we wish to maximize:}
%\begin{align}
\operatorname{Profit} &= \max \left(4x_1 + 2x_2\right)
\end{align}
One idea: graph feasible region, then draw level sets of the objective function,
moving in the direction that seems to increase the profit.
When formerly we considered brute-forcing (uncountably)
infinite points, the problem is reduced to considering corners of the feasible region.

%\begin{align}
%\begin{matrix}
%\max &4x_1 + 2x_2 \\
%\text{subject to} & \begin{aligned}
%x_1 &\geq 0 \\
%x_2 &\geq 0
%\end{aligned}
%\end{matrix}
%\end{align}

\section{Simplex algorithm}
Idea: start at a corner and keep on going to a better neighbor.
But with \(m\) constraints in \(n\) variables, there are \(\begin{pmatrix}
m \\ n
\end{pmatrix}\) corners (solve all systems of \(n\) equations in \(n\) unknowns).

The simplex algorithm also supplies a proof.
If it has achieved objective value \(j\) at some corner,
the coordinates of the corner give a way to combine that corner's
constraints in a way that puts an upper bound on the feasible objective value.

\subsection{Minimizing cost (carpet production)}
There is \emph{demand} \(d_i\) for months \(1\leq i \leq 12\), \(440\leq d_i \leq 920\) for all \(i\).
You have 30 employees that must be paid \(\$2000\)/month.
Each employee can make 20 carpets/month.
An employee can work overtime to make up to 6 extra carpets
per month at a rate of \$180/carpet.
You can fire an employee for \$400 and hire one for \$320.
It costs \$8 to store a carpet for one month.

Define variables \(e_i\) for the number of employees in month \(i\). \(e_0 = 30\). \(x_i\) will be the
number of carpets made in month \(i\). \(s_i\) will be the number of carpets stored by the end of month \(i\).
\(o_i\) is the number of overtime carpets.
\(h_i\) and \(f_i\) are the numbers of the hired and fired workers, respectively.
\begin{align}
	\begin{matrix}
		\min & \sum_i \left(2000 e_i + 180 o_i + 8s_i + 400 f_i + 320 h_i\right) \\
		\text{subj.~to} & \begin{aligned}
		\text{all vars.} &\geq 0 \\
		e_i &= e_{i -1} + h_i - f_i \\
		x_i &= 20e_i + o_i \\
		o_i &\leq 6e_i \\
		s_i &= s_{i - 1} + x_i - d_i
		\end{aligned}
	\end{matrix}
\end{align}

\subsection{Parameterizing LPs}
You can change a minimization to a maximization by negating all variables.
An equality constraint ``\(=\)'' can be replaced by one ``\(\leq\)'' and one ``\(\geq\).''

You can change an inequality to an equality by introducing a slack variable:
\begin{align}
\begin{aligned}
	\sum a_i x_i &\leq b_i
\end{aligned}
&\sim
\left(\begin{aligned}
	\sum a_i x_i + s &= b_i \\
	s &\geq 0
\end{aligned}\right)
\end{align}