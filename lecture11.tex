\chapter{More Greedy Algorithms}
A large text document has 8 different symbols, at differing frequencies:

\noindent
\begin{tabular}{lr}
	Symbol & Frequency \\\hline
	A & 100 \\
	B & 25 \\
	C & 25 \\
	D & 50 \\
	E & 60 \\ 
	F & 40 \\
	G & 200 \\
	H & 500
\end{tabular}

If you encode each letter as 3 bits, then you will need 3000 bits to encode this file. Simply cheaping the popular letter into less bits will not do. But if we encode ``H'' as 0, then we can begin the others with 1 (making them 4 bits) and we can finish the file in 2500 bits.

\section{Prefix-free encoding}
A dictionary is prefix-free if no code is a prefix of any other code.
In prefix-free encoding, the decoding program can be modeled as a binary decision tree
in which the leaves are symbols. The goal is to find a decision tree that specifies the shortest prefix-free
encoding.

\subsection{Optimality conditions}
\begin{enumerate}
	\item In an optimal tree, the least frequent letter is at the greatest depth.
	
	\begin{proof}
		Suppose the least frequent letter is not at the deepest part of the tree but is higher. Then swap it with a more frequent letter that is at the deepest layer of the tree. Now the resultant encoding is shorter.
	\end{proof}
	
	\begin{quotation}
		\em Note: most greedy algorithms will be proved like this:
		prove a property by contradiction in which you can trivially deform an inferior solution into
		a better solution.
	\end{quotation}
	
	\item WLOG, the two least frequent letters are siblings.
	
	\begin{proof}
		Suppose they don't appear together. Swap, preserving encoding length.
	\end{proof}
\end{enumerate}

\subsection{Optimal algorithm}
All letters begin as singleton trees. Combine the two lightest trees until there is only one tree.

\section{Horn SAT}
SAT is boolean satisfiability. Given boolean variables and boolean constraints, find an assignment that satisfies all the constraints. It is in general very hard.

\subsection{Motivation}
There are \(n\) bachelors trying to organize a party. \(x_i\) is true if bachelor \(i\) is invited to the party.
\begin{description}
	\item[unary] some variables must be true/false
	\item[positive clauses/groups] conjunctions on the left implies unary positive on the right.
	\item[negative clauses] negation of a conjunction
\end{description}

\subsection{Greedy algorithm}
Idea: make variables false unless absolutely necessary.

\begin{enumerate}
	\item Begin with all variables false except for unary positive constraints.
	\item Greedily satisfy positive clauses.
\end{enumerate}